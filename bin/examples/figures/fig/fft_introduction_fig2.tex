\documentclass [tikz] {standalone}

\input{header.htex}

\begin {document}
		
		\begin{tikzpicture}
		
		\draw 		(1.5, 0) rectangle (2.5, -5.75);
		\draw 		(12.5, 0) rectangle (13.5, -5.75);
		
		\draw 		(3.25, 0) rectangle (4.25, -2.75);
		\draw 		(3.25, -3) rectangle (4.25, -5.75);
		
		\draw 		(10.75, 0) rectangle (11.75, -2.75);
		\draw 		(10.75, -3) rectangle (11.75, -5.75);
		
		\draw 		(5, 0) rectangle (6, -1.25);
		\draw 		(5, -1.5) rectangle (6, -2.75);
		\draw 		(5, -3) rectangle (6, -4.25);
		\draw 		(5, -4.5) rectangle (6, -5.75);
		
		\draw 		(6.75, 0) rectangle (8.25, -1.25);
		\draw 		(6.75, -1.5) rectangle (8.25, -2.75);
		\draw 		(6.75, -3) rectangle (8.25, -4.25);
		\draw 		(6.75, -4.5) rectangle (8.25, -5.75);
		
		\draw 		(9, 0) rectangle (10, -1.25);
		\draw 		(9, -1.5) rectangle (10, -2.75);
		\draw 		(9, -3) rectangle (10, -4.25);
		\draw 		(9, -4.5) rectangle (10, -5.75);
		
		\draw		(2, -2.875) node[rotate  = 90] {Разделение};
		\draw		(13, -2.875) node[rotate  = 90] {Объединение};
		
		\draw		(3.75, -1.375) node[rotate  = 90] {Разделение};
		\draw		(11.25, -1.375) node[rotate  = 90] {Объединение};
		
		\draw		(3.75, -4.375) node[rotate  = 90] {Разделение};
		\draw		(11.25, -4.375) node[rotate  = 90] {Объединение};	
		
		
		\draw		(5.5, -0.625) node[rotate  = 90] {Разд.};
		\draw		(9.5, -0.625) node[rotate  = 90] {Объед.};
		
		\draw		(5.5, -2.125) node[rotate  = 90] {Разд.};
		\draw		(9.5, -2.125) node[rotate  = 90] {Объед.};
		
		\draw		(5.5, -3.675) node[rotate  = 90] {Разд.};
		\draw		(9.5, -3.675) node[rotate  = 90] {Объед.};
		
		\draw		(5.5, -5.125) node[rotate  = 90] {Разд.};
		\draw		(9.5, -5.125) node[rotate  = 90] {Объед.};
		
		
		\draw		(7.5, -0.625) node[align = center] {ДПФ \\ $N=2$};
		\draw		(7.5, -2.125) node[align = center] {ДПФ \\ $N=2$};
		\draw		(7.5, -3.675) node[align = center] {ДПФ \\ $N=2$};
		\draw		(7.5, -5.125) node[align = center] {ДПФ \\ $N=2$};
		
		\foreach \z in {0,1,...,7}
		{
			\draw[->] (0.75, -\z * 0.75 -0.25) node [left] {$s(\z)$} -- (1.5, -\z * 0.75 -0.25) ;
			\draw[->] (13.5, -\z * 0.75 -0.25)  -- (14.25, -\z * 0.75 -0.25) node [right] {$S(\z)$};	
		}
		
		\foreach \z in {0,1,2,3}
		{
			\draw[red, ->] (2.5, -\z * 0.75 -0.25) -- (3.25, -\z * 0.75 -0.25) ;
			\draw[red, ->] (11.75, -\z * 0.75 -0.25)  -- (12.5, -\z * 0.75 -0.25);			
			
			\draw[blue, ->] (2.5, -\z * 0.75 -3.25) -- (3.25, -\z * 0.75 -3.25) ;
			\draw[blue, ->] (11.75, -\z * 0.75 -3.25)  -- (12.5, -\z * 0.75 -3.25);
			
			
			\draw[red, ->] (6, -\z * 1.5 -0.25) -- (6.75, -\z * 1.5 -0.25) ;
			\draw[blue, ->] (6, -\z * 1.5 -1)  -- (6.75, -\z * 1.5 -1);
			\draw[red, ->] (8.25, -\z * 1.5 -0.25) -- (9, -\z * 1.5 -0.25) ;
			\draw[blue, ->] (8.25, -\z * 1.5 -1)  -- (9, -\z * 1.5 -1);
		}
		
		\foreach \q in {0, 3}
		{
			\foreach \z in {0,1}
			{
				\draw[red, ->] (4.25, -\z * 0.75 -0.25 -\q) -- (5, -\z * 0.75 -0.25 -\q) ;
				\draw[red, ->] (10, -\z * 0.75 -0.25  -\q)  -- (10.75, -\z * 0.75 -0.25 -\q);			
			
				\draw[blue, ->] (4.25, -\z * 0.75 -1.75 -\q) -- (5, -\z * 0.75 -1.75 -\q) ;
				\draw[blue, ->] (10, -\z * 0.75 -1.75 -\q)  -- (10.75, -\z * 0.75 -1.75 -\q);
			

			}
		}
		
		\end{tikzpicture}
\end {document}
		
%		\caption{Разбиение и объединение последовательностей для $N = 8$}\label{fft_introduction:fig2}
%	\end{center}
%\end{figure}


